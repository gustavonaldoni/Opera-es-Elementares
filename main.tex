\documentclass[14pt, aspectratio=169]{beamer}

\usetheme{Copenhagen}
\setbeamertemplate{navigation symbols}{} % Esconde as barras de navegação inferiores
\setbeamercovered{transparent}
\setbeamertemplate{headline}{} % Esconde a barra de navegação no topo do documento, relativa às seções

\newcommand{\R}{\mathbb{R}}
\newcommand{\I}{\mathbb{I}}
\newcommand{\Q}{\mathbb{Q}}
\newcommand{\Z}{\mathbb{Z}}
\newcommand{\N}{\mathbb{N}}

\newcommand{\conj}[1]{\left\{ #1 \right\}}

\newcommand{\skipframe}{\vspace{10.0cm}}

\newtheorem{theo}{Teorema}
\newtheorem{ex}{Exemplo}

\input{pacotes.tex}

\title{Operações elementares}
\subtitle{Pré-cálculo}
\author{Prof. Dr. Márcio Leandro Gonçalves}
\date{\today}
\institute{PUC Minas - Poços de Caldas}

\begin{document}

\begin{frame}
\maketitle 
\end{frame}

\begin{frame}{Sumário}
    \tableofcontents
\end{frame}

\section{Potenciação}

\begin{frame}[allowframebreaks]{Potenciação}

\begin{itemize}
    \item Definição: $a^n = \underbrace{a \cdot a \cdots a}_{n \text{ vezes}} = b$, em que $a, n, b \in \R$.
    \begin{itemize}
        \item $a \rightarrow$ base.
        \item $n \rightarrow$ expoente.
        \item $b \rightarrow$ potência.
    \end{itemize}

    \skipframe

    \item Base positiva $\xrightarrow{\text{gera}}$ potência positiva.

    \begin{itemize}
        \item Exemplo: $ \left( \dfrac{2}{3} \right) = \dfrac{2^4}{3^4} = \dfrac{16}{81}$
    \end{itemize}

    \skipframe

    \item Base negativa:

    \begin{enumerate}[a]
        \item expoente par: potência positiva.

        \begin{itemize}
            \item Exemplo: $(-3)^4 = (-3) \cdot (-3) \cdot (-3) \cdot (-3) = 3^4 = 81$
        \end{itemize}
        
        \item expoente ímpar: potência negativa.

        \begin{itemize}
            \item Exemplo: $\left( -\dfrac{1}{2} \right) = \left( -\dfrac{1}{2} \right) \cdot \left( -\dfrac{1}{2} \right) \cdot \left( -\dfrac{1}{2} \right) = - \left(\dfrac{1}{2} \right)^3 = - \dfrac{1}{8}$
        \end{itemize}
    \end{enumerate}

    \skipframe

    \item Propriedades:

    \begin{enumerate}[a]
        \item Produto de potências de mesma base:
            \begin{equation*}
                a^m \cdot a^n = a^{m + n} \text{, com } a,m,n \in \R.
            \end{equation*}
            
        \item Quociente de potências de mesma base:
        \begin{equation*}
        \label{p2}
            \dfrac{a^m}{a^n} = a^{m - n} \text{, com } a,m,n \in \R \text{ e } a \neq 0. 
        \end{equation*}

        \skipframe

        \item Potência de potência:
        \begin{equation*}
            \left( a^m \right)^n = a^{m \cdot n} \text{, com } a,m,n \in \R.
        \end{equation*}

        \item Potência de um produto:
        \begin{equation*}
            \left( a \cdot b \right)^n = a^n \cdot b^n \text{, com } a,b,n \in \R.
        \end{equation*}

        \skipframe

        \item Potência de um quociente:
        \begin{equation*}
            \left( \dfrac{a}{b} \right) = \dfrac{a^n}{b^n} \text{, com } a,b,n \in \R \text{ e } b \neq 0.
        \end{equation*}

        \item Potência com expoente negativo:
        \begin{equation*}
            \left( a^{-n} \right) = \dfrac{1}{a^n} \text{, com } a,n \in \R \text{ e } a \neq 0.
        \end{equation*}

        \skipframe

        \item Potência com expoente 0:
        \begin{equation*}
            a^0 = 1 \text{, com } a \in \R \text{ e } a \neq 0.
        \end{equation*}

        \begin{proof}
            Seja $a$ e $n$ números reais quaisquer tal que $a \neq 0$. Sabe-se que a expressão $a^n / a^n$ é igual a 1. Usando a propriedade \ref{p2} têm-se que $a^n / a^n = a^{n-n} = a^0$, o que, pela sentença anterior, equivale a 1.
        \end{proof}
        
    \end{enumerate}

    
    
    
\end{itemize}
    
\end{frame}

\section{Radiciação}

\section{Logaritmação}

\section{Exercícios}

\section{Respostas dos exercícios}

\end{document}