\documentclass[14pt, aspectratio=169]{beamer}

\usetheme{Copenhagen}
\setbeamertemplate{navigation symbols}{} % Esconde as barras de navegação inferiores
\setbeamercovered{transparent}
\setbeamertemplate{headline}{} % Esconde a barra de navegação no topo do documento, relativa às seções

\newcommand{\R}{\mathbb{R}}
\newcommand{\I}{\mathbb{I}}
\newcommand{\Q}{\mathbb{Q}}
\newcommand{\Z}{\mathbb{Z}}
\newcommand{\N}{\mathbb{N}}

\newcommand{\conj}[1]{\left\{ #1 \right\}}
\newcommand{\skipframe}{\vspace{10.0cm}}
\newcommand{\parenthesis}[1]{\left( #1 \right)}

\newtheorem{theo}{Teorema}
\newtheorem{ex}{Exemplo}

\input{pacotes.tex}

\title{Operações elementares}
\subtitle{Pré-cálculo}
\author{Prof. Dr. Márcio Leandro Gonçalves}
\date{\today}
\institute{PUC Minas - Poços de Caldas}

\begin{document}

\begin{frame}
\maketitle 
\end{frame}

\begin{frame}{Sumário}
    \tableofcontents
\end{frame}

\section{Potenciação}

\begin{frame}[allowframebreaks]{Potenciação}

\begin{itemize}
    \item Definição: $a^n = \underbrace{a \cdot a \cdots a}_{n \text{ vezes}} = b$, em que $a, n, b \in \R$.
    \begin{itemize}
        \item $a \rightarrow$ base.
        \item $n \rightarrow$ expoente.
        \item $b \rightarrow$ potência.
    \end{itemize}

    \skipframe

    \item Base positiva $\xrightarrow{\text{gera}}$ potência positiva.

    \begin{itemize}
        \item Exemplo: $ \left( \dfrac{2}{3} \right) = \dfrac{2^4}{3^4} = \dfrac{16}{81}$
    \end{itemize}

    \skipframe

    \item Base negativa:

    \begin{enumerate}[a]
        \item expoente par $\xrightarrow{\text{gera}}$ potência positiva.

        \begin{itemize}
            \item Exemplo: $(-3)^4 = (-3) \cdot (-3) \cdot (-3) \cdot (-3) = 3^4 = 81$
        \end{itemize}
        
        \item expoente ímpar $\xrightarrow{\text{gera}}$ potência negativa.

        \begin{itemize}
            \item Exemplo: $\left( -\dfrac{1}{2} \right) = \left( -\dfrac{1}{2} \right) \cdot \left( -\dfrac{1}{2} \right) \cdot \left( -\dfrac{1}{2} \right) = - \left(\dfrac{1}{2} \right)^3 = - \dfrac{1}{8}$
        \end{itemize}
    \end{enumerate}

    \skipframe

    \item Propriedades:

    \begin{enumerate}[a]
        \item Produto de potências de mesma base:
            \begin{equation*}
                a^m \cdot a^n = a^{m + n} \text{, com } a,m,n \in \R.
            \end{equation*}
            
        \item Quociente de potências de mesma base:
        \begin{equation*}
        \label{p2}
            \dfrac{a^m}{a^n} = a^{m - n} \text{, com } a,m,n \in \R \text{ e } a \neq 0. 
        \end{equation*}

        \skipframe

        \item Potência de potência:
        \begin{equation*}
            \left( a^m \right)^n = a^{m \cdot n} \text{, com } a,m,n \in \R.
        \end{equation*}

        \item Potência de um produto:
        \begin{equation*}
            \left( a \cdot b \right)^n = a^n \cdot b^n \text{, com } a,b,n \in \R.
        \end{equation*}

        \skipframe

        \item Potência de um quociente:
        \begin{equation*}
            \left( \dfrac{a}{b} \right) = \dfrac{a^n}{b^n} \text{, com } a,b,n \in \R \text{ e } b \neq 0.
        \end{equation*}

        \item Potência com expoente negativo:
        \begin{equation*}
            \left( a^{-n} \right) = \dfrac{1}{a^n} \text{, com } a,n \in \R \text{ e } a \neq 0.
        \end{equation*}

        \skipframe

        \item Potência com expoente 0:
        \begin{equation*}
            a^0 = 1 \text{, com } a \in \R \text{ e } a \neq 0.
        \end{equation*}

        \begin{proof}
            Seja $a$ e $n$ números reais quaisquer tal que $a \neq 0$. Sabe-se que a expressão $a^n / a^n$ é igual a 1. Usando a propriedade \ref{p2} têm-se que $a^n / a^n = a^{n-n} = a^0$, o que, pela sentença anterior, equivale a 1.
        \end{proof}
        
    \end{enumerate}
\end{itemize}
    
\end{frame}

\section{Radiciação}

\begin{frame}[allowframebreaks]{Radiciação}
    \begin{itemize}
        \item A radiciação é a operação inversa da potenciação.
        \item Por definição, a raiz representa a potenciação como expoente fracionário:

        \begin{equation*}
            \sqrt[n]{x} = x^{\frac{1}{n}} \text{, com } x, n \in \R \text{ e } n \neq 0.
        \end{equation*}

        \item É muito utilizada na obtenção de soluções de equações e na simplificação de expressões aritméticas e algébricas.

        \skipframe

        \item Dados $x \in \R$ tal que $x \geq 0$ e $n \in \N$ com $n \geq 1$, chama-se \emph{raiz enésima} de $x$ o número $y \in \R$, $y \geq 0$ tal que $y^n = x$:

        \begin{equation*}
            \sqrt[n]{x} = y \Leftrightarrow x = y^n
        \end{equation*}

        \begin{itemize}
            \item $n$: índice
            \item $x$: radicando
            \item $y$: raiz
            \item $\sqrt{}$: radical
        \end{itemize}

        \skipframe

        \item Propriedades:

        \begin{enumerate}[a]
            \item $a^{\frac{m}{n}} = \sqrt[n]{a^m} \text{, com } a \in \R, n \in \N^* \text{ e } m \in \Z$.
            \item $\sqrt[n]{a} \cdot \sqrt[n]{b} = \sqrt[n]{a \cdot b} \text{, com } a, b \in \R_+$.  
            \item $\dfrac{\sqrt[n]{a}}{\sqrt[n]{b}} = \sqrt[n]{\dfrac{a}{b}} \text{, com } a, b \in \R_+ \text{ e } b \neq 0$.
            \item $\left( \sqrt[n]{a} \right)^m = \sqrt[n]{a^m} \text{, com } a \in \R_+$.
            \item $\sqrt[n]{\sqrt[m]{a}} = \sqrt[mn]{a} \text{, com } a \in \R_+$.
        \end{enumerate}
    \end{itemize}
\end{frame}

\section{Logaritmação}

\begin{frame}[allowframebreaks]{Logaritmação}
    \begin{itemize}
        \item Os logaritmos facilitam cálculos mais complexos.
        \item Através de suas definições podemos transformar multiplicações em adições, divisões em subtrações, potenciações em multiplicações e radiciações em divisões.

        \skipframe

        \item Dados dois números reais positivos $a$ e $b$, onde $a > 1$ e $b > 0$, existe somente um número real $x$ de modo que:

        \begin{equation*}
            \log_a(b) = x \Leftrightarrow a^x = b
        \end{equation*}

        \begin{itemize}
            \item $a$ = base do logaritmo
            \item $b$ = logaritmando
            \item $x$ = logaritmo de $b$ na base $a$
        \end{itemize}

        \skipframe

        \item Logaritmo é uma função $f: \R_+^* \rightarrow \R$. Os números negativos e o zero não tem logaritmo em $R$.

        \item Logaritmo é a função inversa da função exponencial $y = a^x$.

        \skipframe

        \item Exemplos:

        \begin{enumerate}[a]
            \begin{multicols}{2}
                \item $\log_2 16 = ?$
                    \begin{align*}
                        \log_2 16 &= x \\
                        2^x &= 16 \\
                        2^x &= 2^4 \\
                        x &= 4
                    \end{align*}

                \item $\log_{13} 169 = ?$
                    \begin{align*}
                        \log_{13} 169 &= x \\
                        13^x &= 169 \\
                        13^x &= 13^2 \\
                        x &= 2
                    \end{align*}
            \end{multicols}

        \skipframe
        \end{enumerate}

    \item Propriedades:

        \begin{enumerate}[a]
            \item $a^{\log_a x} = x$ ($\log$ é a função inversa de $a^x$) 
            \item $\log_a 1 = 0$
            \item $\log_a a = 1$

            \skipframe
            
            \item $\log_a (b \cdot c) = \log_a b + \log_a c$
            \item $\log_a \left( \dfrac{b}{c} \right) = \log_a b - \log_a c$
            \item $\log_a b^n = n \cdot \log_a b$
            \item $\log_a \sqrt[n]{b} = \log_a b^{1/n} = \dfrac{1}{n} \cdot \log_a b$
        \end{enumerate}

        \skipframe
             
        \item Logaritmo decimal:

        \begin{itemize}
            \item É aquele cuja base é 10.
            \item São tabelados e também constam nos programas das máquinas de calcular.
            \item Notação: $\log b$ (omite-se a base).
        \end{itemize}

        \skipframe

        \item Exemplos:

        \begin{itemize}
            \item $\log 5 \approx 0,69897$
            \item $\log 50 \approx 1,69897$
            \item $\log 500 \approx 2,69897$
        \end{itemize}

        \item Note que a parte fracionária (\emph{mantissa}) é, no caso, aproximadamente $0,69897$, e a parte inteira (\emph{característica}) é dada por $n - 1$, onde $n \in \N$ e representa o número de algarismos da parte inteira do logaritmando.

        \skipframe

        \item Logaritmo natural (neperiano):

        \begin{itemize}
            \item É aquele cuja base é $e$ (número de Euler).
            \item $e$ é um número irracional, assim como $\pi$ e $\phi$.
            \item $e \approx 2,718281828$.
            \item Notação: $\ln x = \log_e x$
        \end{itemize}
    \end{itemize}
\end{frame}

\section{Exercícios}

\begin{frame}[allowframebreaks]{Exercícios}
    \begin{enumerate}
        \item Simplifique as expressões abaixo:

        \begin{multicols}{2}
            \begin{enumerate}[a]
                \item $\parenthesis{x^2}^3 \parenthesis{x^{-2}}^4$
                \item $\parenthesis{\dfrac{\parenthesis{xy}^5}{2^{-3}}}^{-3}$
                \item $\parenthesis{\dfrac{1}{x^{-2}}} \parenthesis{x^0}^7$
                \item $x^3 y^2 \parenthesis{\dfrac{y}{x^3}} \parenthesis{\dfrac{x^{-1}}{y^3}}$
                \item $\parenthesis{2x}^3 \parenthesis{xy}^2 \parenthesis{\dfrac{y}{x}}^5$
            \end{enumerate}
        \end{multicols}

        \skipframe

        \item Simplifique as expressões abaixo:

        \begin{multicols}{2}
            \begin{enumerate}[a]
                \item $x^{\parenthesis{- \frac{1}{7}}^7} \sqrt{x^8}$
                \item $\sqrt[5]{x^2} x^{\frac{3}{5}}$
                \item $\sqrt[n]{x} \sqrt[n]{y} \dfrac{1}{\sqrt[n]{x}} \dfrac{1}{\sqrt[n]{y}}$
                \item $\parenthesis{\sqrt[3]{x} \sqrt[3]{y}}^3$
                \item $\sqrt[n]{\dfrac{x}{y}} \sqrt[n]{y}$
                \item $\sqrt{\sqrt[3]{\sqrt{x^6}}}$
            \end{enumerate}
        \end{multicols}

        \skipframe

        \item Determine o valor da expressão abaixo:

        \begin{equation*}
             \dfrac{(-5)^2 - 4^2 + \parenthesis{\frac{1}{5}}^0}{3^{-2} + 1}
        \end{equation*}

        \skipframe

        \item Simplifique as expressões:

        \begin{multicols}{3}
            \begin{enumerate}[a]
                \item $\dfrac{3^{n+2} - 3^n}{3^{n+1} + 3^{n-1}}$
                \item $\dfrac{2^{2n + 1} - 4^n}{2^{2n}}$
                \item $\dfrac{2^{n + 1} - 2^{n-2}}{2^n}$
            \end{enumerate}
        \end{multicols}

        \skipframe

        \item Calcule:

        \begin{multicols}{2}
            \begin{enumerate}[a]
                \item $\log 0,01$
                \item $\log 1000$
                \item $2 \log_2 8 - \log_4 16$
                \item $\log_2 \dfrac{\sqrt{32}}{\sqrt[3]{64}}$
                \item $\log_9 3 \sqrt{3}$
                \item $3 \ln e^3 - \ln e^4$
            \end{enumerate}
        \end{multicols}

        \skipframe

        \item Sendo $\log a = 11, \log b = 0,5$ e $\log c = 6$, calcule $x$ sabendo que:
        
        \begin{equation*}
            \log \parenthesis{\dfrac{ab^2}{\sqrt[3]{c}}} = x
        \end{equation*}

        \skipframe

        \item \cite{fme02} Calcule o valor de $S$:

        \begin{equation*}
            S = \log_4 \parenthesis{\log_3 9} + \log_2 \parenthesis{\log_{81} 3} + \log_{0,8} \parenthesis{\log_{16} 32}
        \end{equation*}

        \skipframe

        \item \cite{fme02} As indicações $R_1$ e $R_2$, na escala Richter, de dois terremotos estão relacionadas pela fórmula
        \begin{equation*}
            R_1 - R_2 = \log \parenthesis{\dfrac{M_1}{M_2}}
        \end{equation*}

        em que $M_1$ e $M_2$ medem a energia liberada pelos terremotos sob a forma de ondas que se propagam pela crosta terrestre. Houve dois terremotos: um correspondente a $R_1 = 8$ e outro correspondente a $R_2 = 6$. Calcule a razão $M_1 / M_2$.

        \skipframe

        \item (Mackenzie - SP) Supondo $\log 2 = 0,3$, a raiz da equação $2 - 40^{6x} = 0$ é:

        \begin{multicols}{2}
            \begin{enumerate}[a]
                \item $\dfrac{1}{32}$
                \item 1
                \item 6 
                \item $\dfrac{1}{14}$
                \item $\dfrac{1}{16}$
            \end{enumerate}
        \end{multicols}
        
        
    \end{enumerate}
\end{frame}

\section{Respostas dos exercícios}

\begin{frame}[allowframebreaks]{Respostas dos exercícios}

\begin{multicols}{3}
    \begin{enumerate}
        \item 
        \begin{enumerate}[a]
            \item $\dfrac{1}{x^2}$
            \item $\dfrac{1}{512x^{15}y^{15}}$
            \item $x^8$
            \item $\dfrac{1}{x}$
            \item $8y^7$
        \end{enumerate}

        \item 
        \begin{enumerate}[a]
            \item $x$
            \item $x$
            \item 1
            \item xy
            \item $x^{\frac{1}{n}}$
            \item $x^{\frac{1}{2}}$
        \end{enumerate}

        \item 9
        
        \item 
        \begin{enumerate}[a]
            \item $\frac{12}{5}$
            \item 1
            \item $\frac{7}{4}$
        \end{enumerate}

        \item 
        \begin{enumerate}[a]
            \item -2
            \item 3
            \item 4
            \item $\frac{1}{2}$
            \item $\frac{3}{4}$
            \item 2
        \end{enumerate}

        \item 10

        \item $- \frac{5}{2}$

        \item 100

        \item A
        
    \end{enumerate}
\end{multicols}
\end{frame}

\section{Referências bibliográficas}

\begin{frame}{Referências bibliográficas}
    \bibliography{referencias}
\end{frame}

\end{document}